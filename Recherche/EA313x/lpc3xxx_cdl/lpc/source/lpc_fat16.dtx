@@lpc_fat16.txt
 $Id:: lpc_api.c 4 2007-08-23 00:08:42Z kevinw                       $

   Software that is described herein is for illustrative purposes only 
   which provides customers with programming information regarding the 
   products. This software is supplied "AS IS" without any warranties. 
   NXP Semiconductors assumes no responsibility or liability for the 
   use of the software, conveys no license or title under any patent, 
   copyright, or mask work right to the product. NXP Semiconductors  
   reserves the right to make changes in the software without 
   notification. NXP Semiconductors also make no representation or 
   warranty that such application will be suitable for the specified 
   use without further testing or modification.  
   
       
------------------------------------------------------------------------------

File  : lpc_fat16.c

Date     : 06/05/2002

Purpose  : FAT support library

Notes    : Library is portable. Only simple FAT16 functionality is provided.
           An OS is not required. Operations are 100% blocking. Minimal error
           checking is provided. (Support fot FAT12 can be added with a
           little extra work.)

Issues   : No major bugs observed, when working within the limitations of the
           driver. See the caveats and limitations section at the end of the
           file.

Supporting packages:
           lpc_fat16_private.c - Private FAT16 support functions


How to use this driver:
(Be sure to read the caveats and driver limitations of this driver below)

******************************************************************************
* MAIN DRIVER FUNCTIONS (located in fat16 module)
******************************************************************************
******************************************************************************
* DRIVER INITIALIZATION AND SHUTDOWN
******************************************************************************
DRIVER INITIALIZATION: fat16_init_device
Prior to any call to his driver, the initialization function should be
called. This function binds a device driver to the FAT16 driver,
allocates necessary memory for the device structures, and loads in the
Master Boot Record (MBR) from the device. The driver is designed to be
binded to a device with the functions listed in the fat16_init_device
parameter list.

DRIVER SHUTDOWN: fat16_shutdown
The shutdown function is used to write cached FAT cluster and directory
data back to the device prior to a shutdown. Device structures allocated
for the driver/device will also be released in this function. If any
write operations occured to the device, the cached FAT cluster table
will be written back to the device. It is very important to call this
function prior to shutdown or the device data may not be updated!

******************************************************************************
* PARTITION CONTROL
******************************************************************************
PARTITION STATUS: fat16_get_status
This function can be used to query the states of the 4 partitions. Each
partition can be checked to see the partition type and status. This function
should be used to determine the partition types and statuses prior to
mounting a new parition.

PARTITION SELECTION: fat16_set_partition
This function is used to set the active partition for the device. An active
partition of type FAT16_ can be selected. This function will read in the boot
sector, compute partition gemoetries, and cache the FAT cluster table for
the partition.

******************************************************************************
* FILE MANAGEMENT
******************************************************************************
CREATION OF A FILE DESCRIPTOR: fat16_create_new_file_descriptor
A file descriptor is needed for any file or directory based operations. This
function allocates memory for file descriptor binded to a FAT16 device and
partition. The descriptor is used for all file operations.

DESTRUCTION OF A FILE DESCRIPTOR: fat16_destroy_file_descriptor
A file descriptor is is de-allocated from memory. This function does not
replace fat16_close_file. fat16_close_file does not de-allocate the file
descriptor structure. (See the fat16_close_file function for more info).

ACTIVE DIRECTORY SELECTION: fat16_cd
This function sets the active directory for the file descriptor. All file
read/write operations will occur in this active directory. Note that a file
does not been to be open to use this command - the descriptor only needs to
be available.

CHECK A DIRECTORY ENTRY: fat16_get_dirname
This functions returns the directory entry type, name, and usage flag for the
next directory entry in the active directory. Each successive call to this
function will return the next entry in the directory list, with wraparound
occuring at the end of the list. The entry type is used to determine if the
directory entry is a directory, hidden file, archive file, etc. The usage
flag is used to determine if the directory entry is empty. This function can
be used to compile a list of names in the active directory.

RESET DIRECTORY ENTRY CHECK INDEX: fat16_set_dir_index
This function can be used to reset the directory check index used for the
fat16_get_dirname function.

DELETING A FILE: fat16_delete
This function is used to delete a file in the active directory. Directories
cannot be deleted. Once a file is deleted, it cannot be restored.

OPEN A FILE FOR READING OR WRITING: fat16_open_file
This functions opens a new file for writing or an existing file for reading.
Binary mode is only supported. If a file is opened for write and the files
already exists, the file will be overwritten.

READ DATA FROM A FILE: fat16_read
This function is used to read data from a file. Read operations are
character based operations. The number of bytes read is returned to the
caller. When all the data has been read, the eof flag is set. The file must
have been previously opened for reading with the fat16_open_file function.

WRITE DATA TO A FILE: fat16_write
This function is used to write data to a file. Write operations are
character based operations. The number of bytes written is returned to the
caller. When all the data has been written, a call to fat16_close_file
should be made to make sure all buffered data is written to the device. The
file must have been previously opened for writing with the fat16_open_file
function.

CLOSE AN OPEN FILE: fat16_close_file
This function closes a previously opened file. For read operations, this
just releases the descriptor for another operation. For write operations,
the buffered write data (if any) will be flushed to the device before the
descriptor is released. The descriptor itself is not de-allocated in this
function and is available for another read/write operation as long as it
has not been destroyed.

******************************************************************************
* MEMORY USAGE OF THE FAT16 DRIVER
******************************************************************************
Memory usage for the driver varies depending on the geometry of the device
that has been binded to it. For a Compact FLASH card, the following dynamic
memory allocations are close, but may be different for different devices:
 Binding the driver to a new device         : uses approximately 1KByte (KB)
 Setting an active partition                : uses approximately 16 ~ 130KB
    (Memory required to cache FAT cluster table per device)
 Creating a new file descriptor             : uses approximately 3KB+
    (Memory required to cache active directory and data storage area)

******************************************************************************
* CAVEATS AND DRIVER LIMITATIONS
******************************************************************************
This driver is designed to be a simplistic FAT16 interface driver that does
not require an OS to use. Care must be taken with applications when using
this driver. Specifically, beware of the following:
 - Be very careful when using file write operations, as the active directories
   are cached images. Since each file descriptor has a cached image of the
   active directory, one file descriptor may become 'out of date' with another
   when using write operations. It is not recommended to open more than 1
   file for write operations.
 - When using the fat16_get_dirname function, all directory entry (valid and
   invalid) indexes are returned. If the state of the empty flag is '1', avoid
   use of the entry. Be sure to check the state of the entry type flag also,
   as the only usable flag type is ATTB_ARCHIVE.
 - Be sure to close file descriptors after use. If ANY write operations are
   performed, be sure to call the file descriptor destroy function prior to
   calling the fat16_shutdown function.
 - Be sure to call fat16_shutdown prior to shutting down the filesystem. This
   needs to be done if ANY write has occurred to update the FAT cluster
   table on the device.


