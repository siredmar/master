\section*{Abstract}
\label{sec:Abstract}
Das Thema der Arbeit ist die Analyse und  Entwicklung von Displayschnittstellen für embedded Linux Boards. Dabei werden gängige Video-Schnittstellen untersucht und diese für die Verwendung in eingebetteten Systemen bewertet.\\
Die vorliegende Arbeit beschäftigt sich eingängig mit der Entwicklung zweier Verfahren zur Anzeige von Bilddaten:
\begin{itemize}
\item Möglichkeit für ein ausgewähltes embedded Linux Board zur Anzeige von Bilddaten ohne explizite Hardware-Schnittstelle.
\item Anzeige mittels vorhandenem Grafikchip und HDMI-Anschluss eines ausgewählten embedded Linux Boards.
\end{itemize}
Im Ergebnis der ersten Methode wird deutlich, dass die Entwicklung softwarebasierter Anzeigemethoden einen gewissen Rahmen für schwächere Systeme bieten, jedoch mit einem erheblichen Aufwand in der Softwareentwicklung zu rechnen ist. Sind in einem System bereits Grafikeinheiten in Hardware vorhanden, so ergibt sich der Vorteil des geringeren Rechenaufwands und der hardwarebeschleunigten Methoden zur Anzeige. Der Mehraufwand liegt hier vor allem in der Hardwareentwicklung.
