\chapter{Zitate und Referenzen}
\label{cha:ZitateReferenzen}

Die \NeuerBegriff{Service-orientierte Architektur} (SOA) ist seit einiger Zeit \textit{das} Schlagwort im Bereich der Informationstechnologie. So haben \zB Deutschlands größte Softwarehersteller SAP und die Software AG ihre Unternehmensstrategie komplett auf die SOA ausgerichtet. \Autor{SAP2007} bietet mit \Fachbegriff{Netweaver} seine marktführende ERP-Software auf Basis von SOA an,\footnote{\Vgl\Zitat[S.~127]{SAP2007}} und die \Autor{Software2007b}, die sich selbst als "`The XML Company"' bezeichnet, erweiterte kürzlich noch einmal ihr bereits durchgängig an der SOA orientiertes Produktportfolio durch den Kauf des amerikanischen Unternehmens webMethods um Lösungen zur Unterstützung von Geschäftsprozessen.\footnote{\Vgl\Zitat{Software2007b}} In einem Atemzug mit der SOA werden häufig Webservices genannt, da sie durch ihre hohe Plattformunabhängigkeit und den Einsatz von Internettechnologie oftmals als Referenzimplementierung für die Services in einer SOA angeführt werden. Doch welche Vorteile bietet der Einsatz von Webservices in Unternehmen? Können mit ihnen tatsächlich flexiblere Softwaresysteme entwickelt werden? Und wie einfach ist die Implementierung von Webservices auf unterschiedlichen Plattformen? Diesen Fragen wird sich der Autor in der vorliegenden Arbeit widmen.

Wie bereits in Kapitel \ref{cha:Einleitung} auf Seite \pageref{cha:Einleitung} erwähnt, ist zur Unterstützung von Geschäftsprozessen der Einsatz von Informationstechnologie notwendig. Der Autor verfolgt mit dieser Arbeit das Ziel, einen Geschäftsprozess \todo{Was ist ein Geschäftsprozess?} mit Hilfe von Webservices zu optimieren. Hierzu wird er in diesem Kapitel eine Einführung in das Thema Webservices und die damit in Zusammenhang stehenden Technologien geben, und auch auf mögliche Einsatzbereiche von Webservices im Rahmen der Geschäftsprozessoptimierung eingehen. Tabelle \ref{tab:ElementeDerEreignisgesteuertenProzesskette} auf Seite \pageref{tab:ElementeDerEreignisgesteuertenProzesskette} zeigt ganz tolle Sachen.

Ich empfehle allen Softwareentwicklern die Lektüre von \Zitat{Goodliffe2007}.

\section{Definitionen}
Die Service-orientierte Architektur ist ein Ansatz der Softwareentwicklung, der sich stark am Konzept der Geschäftsprozesse orientiert und mit Hilfe von Webservices implementiert werden kann. In den beiden folgenden Kapiteln werden beide Begriffe eingehend erläutert, worauf in Kapitel \ref{cha:Fazit} die für die Umsetzung von Webservices benötigten Technologien vorgestellt werden.

\section{Service-orientierte Architektur}
\Autor{OASIS2007}\footnote{Die \NeuerBegriff{Organization for the Advancement of Structured Information Standards} ist nach \Zitat{OASIS2007} ein internationales Konsortium aus über 600 Organisationen, das sich der Entwicklung von E-Business-Standards verschrieben hat. Mitglieder sind \zB IBM, SAP und Sun.} definiert den Begriff \NeuerBegriff{Service-orientierte Architektur} (SOA) wie folgt:
\begin{quote}
"`\textbf{Service Oriented Architecture} [\ldots] is a paradigm for organizing and utilizing distributed \textbf{capabilities} that may be under the control of different ownership domains."'\footnote{\Zitat[S.~8]{OASIS2006a}}
\end{quote}
Diese bewusst allgemein gehaltene Definition stammt aus dem Referenzmodell der SOA aus dem Jahr 2006. Dieses Modell wurde mit dem Ziel entwickelt, ein einheitliches Verständnis des Begriffs SOA und des verwendeten Vokabulars zu schaffen, und sollte die zahlreichen bis dato vorhandenen, teils widersprüchlichen Definitionen ablösen.\footnote{\Vgl\Zitat[S.~4]{OASIS2006a}} Dabei wird zunächst noch kein Bezug zur Informationstechnologie hergestellt, sondern allgemein von Fähigkeiten gesprochen, die Personen, Unternehmen, aber eben auch Computer besitzen und evtl. Anderen anbieten, um Probleme zu lösen. Als Beispiel wird ein Energieversorger angeführt, der Haushalten seine Fähigkeit Strom zu erzeugen anbietet.\footnote{\Vgl\Zitat[S.~8f.]{OASIS2006a}}
