\chapter{Einleitung}
\label{cha:Einleitung}

\section{Motivation}
In der heutigen Zeit treten eingebettete Systeme (engl. embedded systems) immer st"arker in den Vordergrund. Gerade in den Bereichen der Industrie, Telekommunikation oder Multimedia w"achst der Bedarf an L"osungen die durch Zuverl"assigkeit, Energiesparsamkeit und kompakter Bauform bestechen.\\
Obwohl eingebettete Systeme meist f"ur den Anwender unsichtbar ihren Dienst verrichten, sind sie doch inzwischen allgegenw"rtig. Im Bereich der Telekomunikation und Unterhaltungselektronik kommt ein solches System im prinzip nicht mehr ohne ein Display aus. Die Moeglichkeit zur Anzeige multimedialer Daten wird zur Kaufentscheidung. Auch hier gilt die Maxime: besser, schneller, groesser.\\
Im Sektor der eingebetten Systeme mit Betriebssystem spielt Linux neben diversen anderen Systemen wie beispielsweise RTOS, OSEK, QNX oder auch Windows eine sehr grosse Rolle. In Verbindung zeigen eingebettete Linuxsysteme mit Displays ein grosses Potential. Mit der beliebten ARM-Architektur lassen sich so kostenguenstige, leisttungsstarke Systeme aufbauen, die die gestellten Aufgaben gut erfuellen kann.


\section{Ziel der Arbeit}
Das Ziel dieser Arbeit ist zu zeigen, dass die Verwendung von Displays mit eingebetten Linux Systemen  je nach Anforderung einfach oder ueber Umwege realisierbar ist. 


\section{Aufbau der Arbeit}
Im ersten Teil der Arbeit werden theoretische Grundlagen gebildet, die fuer das Verstaendnis notig sind. Hier werden Standards wie z.B. HDMI bzw. DVI, LVDS und RGB behandelt. Es wird ein Ueberblick ueber ausgewaehlte embedded Linux Boards gegeben und diese Klassifiziert mit welchen Displayschnittstellen diese ausgestattet sind bzw. ausgestattet werden koennen.
Der Zweite Teil behandelt das embedded Linux Board 'Gnublin', welches von Haus aus keine Displayschnittstelle vorgesehen hat. Hier werden zwei Varianten zur Ansteuerung von Displays erarbeitet. Die Ansteuerung wird hierbei vom Prozessor erledigt, da das 'Gnublin' keine dedizierten Grafikcontroller besitzt.
Im dritten Teil wird fuer leistungsstaerkere embedded Linux-Systeme mit HDMI-Schnittstelle eine Hardware entwickelt, RGB- oder LVDS-Panels anzuschliessen. Um die Displys ueber die entwickelte Hardware anzusteuern, wird der dedizierte Grafikcontroller der Boards verwendet.

\section{Typographische Konventionen}
