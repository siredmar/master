\section{Known Bugs}
Im Laufe der Entwicklung gab es Erschwernisse, welche die Entwicklung verzögerten. 
Als Einschränkung bezüglich des MD050SD, stellt sich die begrenzte Geschwindigkeit von 50~MHz dar. Mit einer maximalen Busgeschwindigkeit von 90~MHz, könnte das Display wesentlich schneller betrieben werden, was die Framerate fast verdoppeln würde. Zusätzlich erscheinen auf dem Display zufällig Artefakte in Form von einzelnen Pixeln, was die Vermutung zulässt, dass die Leitungen von der Adapterplatine oder des MD050SD selbst anfällig für Störungen von außen sein können. \\ \\
Bezüglich dem SSD1289 stellte sich heraus, dass die Verwendung der angebotenen Kommandos aus \reft{tab:Kommandos_SSD1289} nicht für die Adressierung eines RAM-Fensters über mehrere Zeilen zuverlässig funktioniert. Unabhängig vom gesendeten Kommando treten hier zufällige Resets des Displaycontrollers auf, welche das Display nach kurzer Zeit komplett weiß erscheinen lässt. Als Lösung stellte sich die Reservierung einzelner Zeilen dar, die in der Summe das komplette RAM-Fenster abdecken. Nachteilig stellt sich hierbei der erhöhte Adressierungsaufwand dar, da jede Zeile erneut adressiert werden muss.\\ \\
Der Betrieb mit dem SSD1963 stellt sich problematischer dar, als anfangs angenommen. Die Ursache des Problems ist noch ungeklärt, was den Betrieb mit dem SSD1963 derzeit unmöglich macht.
Das Problem stellt sich so dar, dass sich trotz scheinbar korrektem Datenverkehr auf dem 8080-Bus der Displaycontroller nicht initialisieren lässt. Als erster Schritt steht immer die PLL\footnote{PLL: Phase Locked Loop, Phasenregelschleife zur Erzeugung von hohen Taktraten}. Mithilfe der PLL wird der erforderliche Displaytakt von z.~B. 90~MHz erzeugt und der Controller mit dieser Frequenz betrieben. Ein solche schneller Zugriff auf den SSD1963 ist erst nach der Initialisierung der PLL möglich. Bevor dies der Fall ist, kann mit maximal 5 M Words/s \footnote{5M~Word/s: $5*10^6$ Datenwörter pro Sekunde} geschriebenen bzw. gelesen werden (siehe \cite{SSD2008}, S. 72). Der Fehler stellt sich so dar, dass sich die PLL nicht initialisieren lässt. Um den Fehler zu finden, wurden diverse Überlegungen vorangestellt. Angedachte potentielle Fehlerquellen sind
\begin{itemize}
\item zu flache Flanken der Signale
\item 8080-Bus Protokoll nicht eingehalten
\item 8080-Bus Timing nicht im Rahmen der Spezifikationen
\item 8080-Bus Datenverkehr fehlerhaft
\item Leitungsführung auf dem Display selbst schlecht
\end{itemize}
Die Flanken der Signale haben sich nach Messungen mit dem Oszilloskop als nicht zu flach herausgestellt und können als Fehlerquelle ausgeschlossen werden. Die Einhaltung des 8080-Bus Protokolls, samt der Einhaltung der Timings wurden ebenfalls überprüft. Hierzu ist dasselbe Display mit einer funktionierenden Displayansteuerung über die GPIO-Pins aufgebaut worden und jedes Kommando der Initialisierung des SSD1963 mit dem Logic-Analyzer aufgenommen worden. Derselbe Displaytreiber, mit dem Unterschied der Ansteuerung über das SRAM-Interface, wurde ebenfalls aufgezeichnet und mit den Daten der vorhergehenden Messung verglichen. Diese Methode schließt einen Fehlerhaften Datenverkehr aus und lässt zusätzlich die Rahmenbedingungen für das 8080-Interface selbst und dessen Timing überprüfen. Für die Aufzeichnung, wurde der User-Space-Treiber dahingehend modifiziert, dass er vor dem Senden die Bestätigung des Anwenders abfragt. Die Abbildungen \ref{fig:ssd1963_gpio} und \ref{fig:ssd1963_sram} zeigen einen exemplarischen Datentransfer für die beiden Ansteuerungsmethoden. Zu erkennen ist, dass dasselbe Wort an den Datenpins D[7:0] anliegt, und die Steuersignale CS, WR, RD und A15 entsprechend innerhalb der markierten Zonen entsprechend dem 8080-Interface geschaltet werden. 

\begin{figure}[htp]
        \begin{center}
        \begin{subfigure}[htp]{1\textwidth}
			%\begin{figure}[h!]
			\centering
			\fbox{	\includegraphics[width=1\textwidth]{TeilA/print_34_dip.png}}
	\caption{SSD1963 mit GPIO}
			\label{fig:ssd1963_gpio}
		\end{subfigure}


        \begin{subfigure}[htp]{1\textwidth}
%\begin{figure}[h!]
	\centering
\fbox{	\includegraphics[width=1\textwidth]{TeilA/print_34_ext.png}}
	\caption{SSD1963 mit SRAM-Interface}
	\label{fig:ssd1963_sram}
\end{subfigure}

		\end{center}
\caption{SSD1963: Vergleich GPIO- und SRAM-Ansteuerung}
	\label{fig:ssd1963_gpio_sram}
\end{figure}
\newpage
Das geforderte Timing ist in \refa{fig:ssd1963_timing_constraints} zu sehen und beinhaltet die minimal notwendigen Zeiten zwischen den einzelnen Signalen.
\begin{figure}[htp]
%\begin{minipage}[t]{0.8\textwidth}
%\begin{figure}[h]
	\centering
\fbox{	\includegraphics[width=1.0\textwidth]{TeilA/ssd1963_writeCycleConstraings.png}}
	\caption{8080-Timingbedingung für SSD1963}
	\label{fig:ssd1963_timing_constraints}
\end{figure}
Diese Mindestzeiten wurden innerhalb der Oszilloskopbilder eingehalten und verifiziert, sodass ein Fehler mit dem Protokoll des 8080-Interface ausgeschlossen werden kann. Bezüglich der uninitialisierten PLL des Displays und der verringerten Schreibrate ist in \refa{fig:ssd1963_sram} eine Chip-Select-Laenge von 406~Nanosekunden erkennbar, was einer Schreibgeschwindigkeit von 2.46~MHz entspricht. Dies ist weit unter den geforderten 5~MHz im uninitialisierten Zustand. Ein zu schnelles Schreiben ist in diesem Fall ebenfalls ausgeschlossen. Nachdem verifiziert wurde, dass aus der Adapterplatine vom Gnublin Extended die richtigen Signale geliefert werden, bleibt als Ursache nur noch das Display selbst. Da die Leitungsführung des ursprünglich verwendeten 4.3" Displays nicht optimal ist, bei dem die 8080-Leitungen quer über die Platine geführt und im Anschluss von den RGB-Signalen im 90 Grad Winkel gekreuzt werden, wurde der Fehler im schlechten Platinendesign des Displays gesucht. Aufgrund dessen fand das 5" Display mit demselben Controller Verwendung, das eine optimierte Leitungsführung besitzt. Jedoch waren die aufgeführten Lösungsansätze nicht zielführend, weswegen das Display nicht mit dem Gnublin Extended unter Verwendung des SRAM-Interface einsetzbar ist.