\chapter{Zusammenfassung}
\label{cha:Zusammenfassung}
Die Ziele der beiden Teile dieser Masterarbeit waren
\begin{itemize}
\item Optimierte Portierung des vorausgehenden Projekts zur Ansteuerung von TFT-Displays
\item Ausnutzung der vollen Grafikleistung von Linux-Boards mit Grafikhardware über die HDMI-Schnittstelle durch eine selbst entwickelte Hardware zur Ansteuerung von Displays
\end{itemize}
Im Teil A dieser Arbeit ist auf die Low-Level Programmierung des verwendeten Prozessors \code{LPC3131} eingegangen worden. Es wurde ein Verfahren entwickelt, Displays mit 8080-Interface an einem Speicherbus des Prozessors zu betreiben. Diese Methode findet in einem entwickelten Framebuffer-Treiber im Linux-Kernel, einem User-Space-Treiber sowie im Bootloader Verwendung.\newline

Hierbei ergaben sich Erschwernisse in der Entwicklung, die teilweise gelöst wurden aber noch Fragen bzgl. der Fehlerursachen offen lassen. 
So scheint die Verwendung der Grafikcontroller \code{SSD1963} im Vergleich zum \code{SSD1289} und dem Display \code{MD050SD} in der entwickelten Anwendung nicht nutzbar zu sein. Die Fehlerursache konnte trotz ausgedehnter Suche nicht ermittelt werden.\newline
Das Ziel der optimierten Ansteuerung unter Verwendung des Speicherinterface zeigt sich mit dem \code{MD050SD} als erfolgreich. Es wird deutlich, dass es auch für leistungsschwache Systeme gute Möglichkeiten zur Anzeige gibt.\newline

Mit der Ausnutzung der Grafikhardware des verwendeten \code{Raspberry Pi} ist dieses als Einheit zu sehen, welche über die HDMI-Schnittstelle mit der Außenwelt kommuniziert. Die berechneten Video-Signale werden von einer Onboard-Grafikeinheit zur Verfügung gestellt, die mit der entwickelten Hardware aus Teil B aufgegriffen und ausgewählte TFT-Displays angeschlossen werden können. Die Schnittstellen zum Anschluss dieser Displays sind der \code{RGB}-Bus sowie ein \code{LVDS}-Interface.\newline
Um einen Plug-And-Play-Betrieb an einer HDMI-Quelle zu ermöglichen, wurden EDID-Daten auf der Platine hinterlegt und die Möglichkeit eröffnet, diese mittels einer Verbindung über USB mit einem Computer neu zu beschreiben.\newline
Während der Entwicklung sind ebenfalls Fehler aufgetreten, die teilweise behoben wurden. In der gefertigten Hardware sind diese als Fehler im Schaltplan zu finden.\newline
Dennoch ist das Ziel der Ausnutzung der Grafikhardware dahingehend erfüllt, dass eine funktionierende Methode entwickelt wurde, die HDMI-Signale anzuzeigen.