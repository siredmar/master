% ä ü ö ß 

\chapter{Einleitung}
\label{cha:Einleitung}

\section{Motivation}
In der heutigen Zeit treten eingebettete Systeme (engl. embedded systems) immer stärker in den Vordergrund. Gerade in den Bereichen der Industrie, Telekommunikation oder Multimedia wächst der Bedarf an Lösungen die durch Zuverlässigkeit, Energiesparsamkeit und kompakter Bauform bestechen.\\
Obwohl eingebettete Systeme meist für den Anwender unsichtbar ihren Dienst verrichten, sind sie doch inzwischen allgegenwärtig. Im Bereich der Telekommunikation und Unterhaltungselektronik kommt ein solches System im Prinzip nicht mehr ohne ein Display aus. Die Möglichkeit zur Anzeige multimedialer Daten wird zur Kaufentscheidung. Auch hier gilt die Maxime: besser, schneller, größer.\\
Im Sektor der eingebetten Systeme spielen Betriebssystem wie Linux neben diversen anderen Systemen wie beispielsweise RTOS, OSEK, QNX oder auch Windows eine sehr große Rolle. In Verbindung mit Displays zeigen eingebettete Linuxsysteme ein großes Potential. Mit der beliebten ARM-Architektur lassen sich kostengünstige, leistungsstarke Systeme aufbauen, welche die gestellten Aufgaben gut erfüllen können. Sieht man sich allein den Marktanteil von Smartphones welche auf Android-Basis arbeiten an, wird der Trend klar, dass Hersteller eine offene Basis bevorzugen (siehe \cite{android2014} und \cite{Brandt2013}).\\
Es scheint ersichtlich, dass auch in Zukunft Linux auf eingebetteten Systemen eine immer größere Rollen spielen wird. 

\section{Ziel der Arbeit}
Dieser Arbeit vorausgehend stand eine Projektarbeit, bei der ein TFT-Display mittels GPIO-Pins mit dem Einplatinenrechner \code{Raspberry PI} angesteuert wurde. Das Ziel diese Arbeit ist an der Idee anzuknüpfen und diese Methode auf eine leistungsschwächere Plattform zu portieren sowie das Ausschöpfen der vollen Grafikleistung von Linux-Boards mit Grafikhardware und HDMI-Schnittstelle bezüglich einer selbst entwickelten Anzeigemöglichkeit.
\section{Aufbau der Arbeit}
Im ersten Teil der Arbeit werden theoretische Grundlagen gebildet, die für das Verständnis nötig sind. Hier werden diverse standardisierte Video-Schnittstellen behandelt. Es wird ein Überblick und Klassifizierung über ausgewählte embedded Linux Boards geschaffen.
Der Zweite Teil behandelt das embedded Linux Board \code{Gnublin Extended}. Hier werden zwei Varianten zur Ansteuerung von Displays erarbeitet. Die Ansteuerung wird hierbei vom Prozessor erledigt, da das \code{Gnublin} keine dedizierten Grafikcontroller besitzt.
Im dritten Teil wird für leistungsstärkere embedded Linux-Systeme mit HDMI-Schnittstelle eine Hardware entwickelt, die es ermöglicht RGB- oder LVDS-Panels anzuschließen. Um die Displays über die entwickelte Hardware anzusteuern, wird der dedizierte Grafikcontroller der Boards verwendet. Jeweils am Ende der beiden großen Kapiteln findet sich eine Sektion mit bekannten Fehlern und Problemen bei der Entwicklung. Die zum Schluss kommende Zusammenfassung rundet die Arbeit ab und wirft einen Blick auf die Arbeit im Rück- und Ausblick.  

\section{Typographische Konventionen}
Werden in dieser Arbeit Teile des Textkörpers im diesem Stil z. B. \code{Textbaustein} geschrieben, so handelt es sich hierbei um:
\begin{itemize}
\item Softwarekomponenten
\item Funktionsnamen
\item Variablen
\item Signalnamen
\item Registerbezeichnungen
\item Bauteilbezeichnungen
\item Modulbezeichnungen von Bauteilen
\item ...
\end{itemize}
Werden Abkürzungen genannt, so sind diese in einer Fußnote auf derselben Seite beschrieben.
Sofern nicht anders gekennzeichnet, sind alle Quellcodes in der Programmiersprache C geschrieben.
\section{Verwendete Programme}
Um Schaltpläne und Layouts zu erstellen, wurde das Programm Eagle von Cadsoft\footnote{\url{http://www.cadsoft.com/}} verwendet. Im Rahmen von Teil B dieser Arbeit ist eine Bauteilbibliothek entstanden, um alle benötigten Bauteile im Schaltplan und Layout verwenden zu können. Diese Bibliothek befindet sich im Anhang auf der CD.
Um 3D Bilder von Platinenlayouts zu erzeugen, wurde das Eagle Plugin Eagle3D\footnote{\url{http://sourceforge.net/projects/eagle3d.berlios/}} verwendet.\\
Für elektrische Simulationen wurde das Programm LTSpice\footnote{\url{http://www.linear.com/designtools/software/}} von Linear Technology verwendet. Die für den Teil B durchgeführten Simulationen befinden sich im Anhang auf der CD.\\
Zur Entwicklung der Programme für die Plattformen PC, ARM und AVR\footnote{AVR: Atmel ATMEGA Prozessor, Akronym für: \textbf{A}lf (Egil Bogen) and \textbf{V}egard (Wollan)'s \textbf{R}ISC processor} wurde Eclipse\footnote{\url{https://www.eclipse.org/}} verwendet. Die verwendeten Compiler sind allesamt Plattformabhängige gcc-Versionen\footnote{\url{https://gcc.gnu.org/}}. \reft{tab:verwendete_compiler} zeigt eine Übersicht der verwendeten Compiler für diese Arbeit.

\begin{table}[h]
\begin{tabular}{|p{4.5cm}|p{4cm}|p{4cm}|}\hline
\rowcolor{TableBackgroundColor} 
\textbf{Plattform}		&	\textbf{Compiler}		&	\textbf{Version}  \\ \hline
 Linux 3.10.11-smp i686	&	gcc						& 4.8.1	\\ \hline
 Atmel ATMega88p		&	avr-gcc					& 4.3.3	\\ \hline
 ARM9 NXP LPC313x		&	arm-linux-gnueabi-gcc	& 4.6.4	\\ \hline
\end{tabular}
\caption{Verwendete Compiler}
\label{tab:verwendete_compiler}
\end{table}